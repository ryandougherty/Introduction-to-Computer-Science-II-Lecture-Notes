\newsection{Heaps}
%
In this section we will introduce the heaps data structure. It is a special type of binary search tree that is used for many data structures.

\subsection*{Heaps}
A heap is defined as a binary search tree such that it has the following properties:
\begin{itemize}
\item The tree is complete: this means that each row of the tree is completely full except for possibly the last row.
\item There is some constraint on the elements.
\end{itemize}
For example, the constraint for adding an element to the heap is that the two child nodes are less than or equal to the parent. This example is called a ``max" heap (i.e., the root is the maximum element in the heap).

\subsection*{Heap Operations}
Heaps support the following operations:
\begin{itemize}
\item Adding an element to the heap.
\item Finding the maximum value in the heap.
\item Removing the maximum value in the heap.
\end{itemize}
We now detail each of these operations and how they work.

\subsection*{Adding an Element}
To add an element $x$ to a heap, we do the following:
\begin{itemize}
\item Add $x$ as a leaf such that the heap is still complete.
\item Move $x$ toward the root, exchanging positions with its parent, until the heap property says that $x$ cannot move up the heap any more.
\end{itemize}

\subsection*{Removing Largest Element}
To remove the largest element from a heap, we do the following (assuming we have a max heap):
\begin{itemize}
\item Remove the root and reconstruct heap from the 2 disjoint sub-heaps.
\item Move the last leaf of the heap to be the new heap of the tree.
\item Move this new root down the heap as needed until the heap property says that it cannot move down the heap any more.
\end{itemize}

\subsection*{HeapSort}
A while ago we covered various sorting algorithms. There actually is a very fast algorithm for sorting using the heap data structure. The way that it works (informally) is that it adds each element from a list into a heap, and then removes them one at a time. The largest element comes off the heap first, and then after re-``heapifying" (fixing the heap), the entire sequence of removing elements will be in descending order.

\par The following is how the algorithm works:
\begin{itemize}
\item Build a heap from the array (as a max-heap) of size $n$.
\item Repeat the following $n$ times: extract the root, put it aside.
\item The order of extraction is a sorted array, in ascending order. If we instead wanted an array in descending order, we use a min-heap.
\end{itemize}

\par The key to how this algorithm works is the re-``heapifying" step. The following is how that algorithm works:
\begin{itemize}
\item Repeat the following until both children are larger or are at the bottom of the heap:
\item Compare the node and its children. If at least one of them is larger than the node, swap it with the larger child; otherwise, we terminate. 
\item If we don't terminate, we go back to the first step with the node's new location.
\end{itemize}

\subsection*{HeapSort in Java}
The following is an implementation of HeapSort in Java, which uses many of the concepts described above.

\begin{lstlisting}
public class HeapSorter {
     private int[] a;
     public HeapSorter(int[] anArray) {
          a = anArray;
     }   
     // Sorts the array managed by this heap sorter.
     public void sort() {
          int n = a.length - 1;
          // for each node that has at least one child node
          for (int i = (n - 1) / 2; i >= 0; i--)
               fixHeap(i, n);
          while (n > 0) {
               swap(0, n);
               n--;
               fixHeap(0, n);
           }
     }
     private void swap(int i, int j) {
          int temp = a[i];
          a[i] = a[j];
          a[j] = temp;
     }
     private void fixHeap(int rootIndex, int lastIndex) {
          int rootValue = a[rootIndex];   //Remove root
          // Promote children while they are larger than the root
          int index = rootIndex;
          boolean more = true;
          while (more) {
               int childIndex = getLeftChildIndex(index);
               if (childIndex <= lastIndex) {
                    // Use right child instead if it is larger
                    int rightChildIndex = getRightChildIndex(index);
                    if (rightChildIndex <= lastIndex &&
                         a[rightChildIndex] > a[childIndex]) {
                         childIndex = rightChildIndex;
                    }
                    if (a[childIndex] > rootValue) {
                         // Promote child
                         a[index] = a[childIndex];
                         index = childIndex;
                    } else {
                         // Root value is larger than both children
                         more = false;
                    }
               } else {
                    // No children
                    more = false; 
               }
          }
          // Store root value in vacant slot
          a[index] = rootValue;
     }
     private static int getLeftChildIndex(int index) {
          return 2 * index + 1;
     }
     private static int getRightChildIndex(int index) {
          return 2 * index + 2;
     }
}
\end{lstlisting}
A straightforward analysis shows that the worst-case running time of HeapSort is $O(n \times log(n))$ for an array of size $n$.