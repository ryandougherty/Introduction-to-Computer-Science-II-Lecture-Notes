\newsection{Interfaces, Abstraction, \& Polymorphism}
%
In this section we will start the ``real" part of this second set of lecture notes. This particular section deals with some object-oriented design principles, and Java supports them very well.

\subsection{Interfaces}
An interface has the same structure as a class, but instead has the \verb|interface| keyword where \verb|class| would go, and only consists of two things: method declarations (the signature of the method plus a semicolon), and constants (with the \verb|final| keyword). An interface is a ``contract" of sorts, where some normal class will ``implement" (via the \verb|implements| keyword). By doing this, that particular class must implement every single method listed in the interface - otherwise, it is a compiler error. For example, if we created an interface for a \verb|Ship|:
\begin{lstlisting}
public interface ShipInterface {
     public void shoot(); // just a declaration, no implementation
}
\end{lstlisting}
We can then implement this interface in any class we see fit:
\begin{lstlisting}
public class Ship implements ShipInterface {
     // ...
     // not including shoot() will have a compiler error
     public void shoot() {
          System.out.println("This ship did shoot!");
     }
     // ...
}
\end{lstlisting}
We can also implement multiple interfaces at once as comma-separated values:
\begin{lstlisting}
public class A implements Interface1, Interface2, Interface3 {
     // ...
}
\end{lstlisting}

\subsection{Inheritance}
Inheritance is a well-known technique in object-oriented programming to organize a lot of similar code in a hierarchical way. In this system, there are two kinds of classes: $children$ (or $base$), and $parent$ (or $super$) classes. The way it works is that the child classes inherit some information from the parent class. For Java, a parent can have one or more children, but a child can have at most one parent. Also, one may only inherit from a class if it is not declared as \verb|final|.
\\ \\
A child class that inherits from a parent class gets all of the \verb|public| methods, but none of the \verb|private| methods (it does get \verb|private| instance variables, however). So how do we enforce encapsulation for child classes then? This is where the \verb|protected| keyword comes in. It has the same behavior as \verb|private|, but any classes that inherit from it can use it. It is somewhat more ``free" than \verb|private|, but also has encapsulation.
\\ \\
In Java, in order to inherit from a class, one must use the \verb|extends| keyword:
\begin{lstlisting}
public class A {
     // ...
}
public class B extends A {
     // ...
}
\end{lstlisting}
A way that inheritance is useful is for allowing code reuse for child classes. For example, one may use the \verb|super| keyword to denote the parent method/constructor to call, and it calls the appropriate method/constructor in the parent class:
\begin{lstlisting}
public class A {
     public A() {
          // ...
     }
}
public class B extends A {
     public B() {
          super(); // calls A();
          // ...
     }
}
\end{lstlisting}
Also, for all of the instance variables declared in the parent class, one does not need to re-declare them in child classes. However, one must use the \verb|this| modifier to do so:
\begin{lstlisting}
public class A {
     private int a;
     public A() {
          a = 0;
     }
}
public class B extends A {
     public B() {
          this.a = 1; // must use this modifier
     }
}
\end{lstlisting}

\subsection{Abstract}
Another keyword in Java is the \verb|abstract| keyword. It is used for a class (that now has the \verb|abstract| modifier) that says that ``this method will be implemented in child classes, but I won't personally implement it." The \verb|abstract| method in the parent class is just a method header (like would be in an interface, but now has the \verb|abstract| modifier), and is a normal method in child classes. Here is an example:
\begin{lstlisting}
public abstract class Person {
     public abstract double computePay();
}
public class CEO extends Person { // no abstract
     public double computePay() { // no abstract
          return 100000.0;
     }
}
\end{lstlisting}

\subsection{Polymorphism}
Now we get to the most important part of this section - polymorphism. It is a way, at runtime, for deciding which method to call when there is a hierarchy of classes, and is relevant to inheritance. For example, if we have the classes:
\begin{lstlisting}
public class A {
     public void m() { ... }
}
public class B extends A {
     public void m() { ... }
}
public class C extends A {
     public void m() { ... }
}
\end{lstlisting}
We can develop code that has variables that can be one or more of these different classes without having to rewrite code:
\begin{lstlisting}
A a1 = new A(); // works
B b1 = new B(); // works
C c1 = new C(); // works
A a2 = new B(); // ok, B has all attributes of A
// B b2 = new A(); // not ok, A does not have attributes of B
a2.m(); // calls B.m() even though a2 has declared type A
a1.m(); // calls A.m() because a1 is of type A
\end{lstlisting}

\subsection{Written Exercises}
\setcounter{counter}{1}
\begin{enumerate}[label={\arabic{counter}\addtocounter{counter}{1}}.]
\item What is the meaning of declaring a class \verb|abstract| and \verb|final|? What do you think will happen?
\end{enumerate}