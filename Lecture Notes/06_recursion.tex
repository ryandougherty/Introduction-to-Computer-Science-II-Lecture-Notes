\newsection{Recursion}
%
In this section we will introduce the concept of recursion. Recursion is all about reducing source code complexity by calling the method in which the statements occur with a ``smaller" parameter. We will give a motivating example: computing the factorial of an integer. 
\\ \\
For recursion to work, we need two key factors: base case(s), and sub-problem(s). The base case(s) is/are what define when the recursion terminates, and handle the ``simple" examples. The sub-problem(s) are all about assuming that the problem of the size given by the sub-problem(s) is/are solved. Then, all we need to provide is the code to move from the sub-problem to the initial problem. Choosing an appropriate sub-problem size is problem-specific, but common examples include decreasing the input by 1, or dividing by 2.

\subsection{Example: Factorial}
The factorial of an integer, written $x!$, is precisely $1 \times 2 \times 3 \times ... \times (x-1) \times x$. We could write a simple \verb|for|-loop, as follows:
\begin{lstlisting}
public int factorial(int x) {
     int xCopy = x;
     int result = 1;
     while (xCopy >= 1) {
          result *= xCopy;
          xCopy--;
     }
}
\end{lstlisting}
But there are too many cases for the program to give the wrong output.
\\ \\
For example, the base cases for factorial are: $0! = 1$ and $1! = 1$. An appropriate sub-problem for factorial is $(x-1)!$. Moving from $(x-1)!$ to $x!$ is easy: all we need to do is multiply by $x$, since $(x-1)! \times x = x!$. Since we have both the base cases and the sub-problem, we can now define a recursive version of factorial:
\begin{lstlisting}
public int factorialRecursive(int x) {
     // handle base cases first
     if (x == 0 || x == 1) { // can also use x <= 1
          return 1;
     } else { // handle subproblem step
          return factorialRecursive(x-1) * x;
     }
}
\end{lstlisting}
As one can see, if we did not have a base case, then the method would keep calling itself over and over without terminating. Therefore, when doing recursion, \textbf{always have a base case}. 

\subsection{Example: Fibonacci Numbers}
The Fibonacci numbers are defined as follows: $F_0 = 1, F_1 = 1, F_n = F_{n-1} + F_{n-2}$ for $n \ge 2$. The first few Fibonacci numbers are 1, 1, 2, 3, 5, 8, 13, and so forth. Informally, each number is the sum of the previous two. Again, we could create a \verb|for|-loop to find the $n$th Fibonacci number, but we can simplify with recursion.
\\ \\
So what are the base cases? They were given to us: $F_0 = 1, F_1 = 1$. So what is the subproblem? We actually have two here: $F_{n-1}$, and $F_{n-2}$. To get to $F_n$, all we need to do is to add $F_{n-1}$ and $F_{n-2}$. Therefore, we can easily create a recursive version for the Fibonacci numbers:
\begin{lstlisting}
public int fibonacciRecursive(int n) {
     if (n == 0 || n == 1) {
          return 1;
     } else {
          return fibonacciRecursive(n-1) + 
                 fibonacciRecursive(n-2);
     }
}
\end{lstlisting}

%\subsection{Written Exercises}
%\setcounter{counter}{1}
%\begin{enumerate}[label={\arabic{counter}\addtocounter{counter}{1}}.]
%\item What is the big-O notation for the operations on a \{linked list, stack, queue\}?
%\item What do you think is the big-O notation for cost of inserting an element into an array/\verb|ArrayList|? Assume, if needed, constructing a new array/\verb|ArrayList| takes $O(1)$ time.
%\end{enumerate}